The following content outlines a typical section structure. It is intended to show you how the structure may look like including tables, pictures, and several examples of info boxes. Note also the glossary entries, such as \gls{bm25}~\cite{RobertsonZ09}, or math glossary links, such as \gls{mbm25-math}. Those are linked to the glossary in the PDF. All colors are defined in the \texttt{preamble.tex} file. Feel free to change them. There are two color modes, the non-printing and the printing mode. The printing mode is intended to reduce the number of colored pages (e.g., the vertical line on each page next to the page number will be gray rather than blue in print mode) but keep the highlights (e.g., info boxes) colored.

\section{Elements of This Template}\label{sec:intro}

Here is an example of cool tables.

\begin{table}[ht]
\caption{Example table with math and verbatim texts.}
\label{tab:jacobi}
\centering
\renewcommand{\arraystretch}{1.5}
\begin{tabular}{c:l}\hline
\textbf{System} & \hspace{0.1cm}\textbf{Representation} \\\hline
\cellcolor{TableRowColor}Rendered Version & \cellcolor{TableRowColor}\hspace{0.1cm}$P_n^{(\alpha, \beta)}(\cos(a\Theta))$ \\
Generic \LaTeX & \hspace{0.1cm}\small\verb|P_n^{(\alpha, \beta)}(\cos(a\Theta))| \\
\cellcolor{TableRowColor}Semantic \LaTeX & \cellcolor{TableRowColor}\hspace{0.1cm}\small \verb|\JacobipolyP{\alpha}{\beta}{n}@{\cos@{a\Theta}}| \\
Maple & \hspace{0.1cm}\small\verb|JacobiP(n, alpha, beta, cos(a*Theta))| \\
\cellcolor{TableRowColor}Mathematica & \cellcolor{TableRowColor}\hspace{0.1cm}\small \verb|JacobiP[n, \[Alpha], \[Beta], Cos[a \[CapitalTheta]]]| \\
SymPy & \hspace{0.1cm}\footnotesize\verb|jacobi(n,Symbol('alpha'),Symbol('beta'),cos(a*Symbol('Theta')))| \\\hline
\end{tabular}
\end{table}

\begin{table}[ht]
\caption[Short caption for LOTs.]{Table with colorful ticks and crosses.}
\label{tab:tex-cas-translations}
\centering
\renewcommand{\arraystretch}{1.3}
\begin{tabular}{l;{1pt/1pt}c:c:c:c:c:c}\hline
	\multicolumn{1}{c:}{\textbf{\LaTeX}} & \textbf{Rendering} & \texttt{MM} & \texttt{SP} & \texttt{ST} & \texttt{WA} & \texttt{LCT}\textsubscript{1} \\\hline
	\rowcolor{TableRowColor}
	\verb|\int_a^b x dx| 			& $\int_a^b x dx$ 			& \crossMark & \checkMark & \crossMark & \checkMark & \crossMark \\
	\verb|\int_a^b x \mathrm{d}x| 	& $\int_a^b x \mathrm{d}x$ & \crossMark & \crossMark & \crossMark & \checkMark & \crossMark \\
	\rowcolor{TableRowColor}
	\verb|\int_a^b x\, dx| 		& $\int_a^b x\, dx$ 		& \checkMark & \checkMark & \crossMark & \checkMark & \crossMark \\
	\verb|\int_a^b x\; dx| 		& $\int_a^b x\; dx$ 		& \crossMark & \checkMark & \crossMark & \checkMark & \crossMark \\
	\rowcolor{TableRowColor}
	\verb|\int_a^b x\, \mathrm{d}x|& $\int_a^b x\, \mathrm{d}x$& \crossMark & \crossMark & \crossMark & \checkMark & \crossMark \\
	\verb|\int_a^b \frac{dx}{x}| 	& $\int_a^b \tfrac{dx}{x}$ & \crossMark & \checkMark & \crossMark & \checkMark & \crossMark \\
	\rowcolor{TableRowColor}
	\verb|\sum_{n=0}^N n^2| 		& $\sum_{n=0}^N n^2$ 		& \checkMark & \checkMark & \checkMark & \checkMark & \checkMark \\
	\verb|\sum_{n=0}^N n^2 + n| 	& $\sum_{n=0}^N n^2 + n$ 	& \questionMark & \questionMark & \crossMark & \questionMark & \questionMark \\
	\rowcolor{TableRowColor}
	\verb|{n \choose m}| 			& $\binom{n}{m}$ 			& \crossMark & \crossMark & \crossMark & \checkMark & \crossMark \\
	\verb|\binom{n}{m}| 			& $\binom{n}{m}$ 			& \checkMark & \checkMark & \checkMark & \checkMark & \checkMark \\
	\rowcolor{TableRowColor}
	
	{\footnotesize\verb|P_n^{(\alpha,\beta)}(\cos(a\Theta))|} & $P_n^{(\alpha,\beta)}(\cos(a\Theta))$ & \checkMark & \crossMark & \crossMark & \crossMark & \checkMark \\
	\verb|\cos(a\Theta)| 			& $\cos(a\Theta)$ 			& \checkMark & \checkMark & \checkMark & \checkMark & \checkMark \\
	\rowcolor{TableRowColor}
	\verb|\frac{d}{dx} \sin(x)| 			& $\frac{d}{dx} \sin(x)$ 			& \crossMark & \checkMark & \crossMark & \checkMark & \crossMark \\
	\hline
\end{tabular}
\end{table}

And a figure.

\begin{figure}[t]
	\centering
	\includegraphics[width=\textwidth]{moi-layers-4L.pdf}
	\caption[Short caption for the List of Figures.]{Long caption directly below the picture.}
	\label{fig:moi-layers-4l}
\end{figure}

\pagebreak
And a lot of different information boxes. Take a look at \verb|boxexample.tex| for further details.

\begin{thesisobjectivebox}
Your thesis objective
\end{thesisobjectivebox}

\begin{researchtaskbox}{Research Tasks Box}
\begin{enumerate}[label=\textbf{\Roman*}, labelsep=1em]
\setlength\itemsep{0.25em}
\item\label{rt:I} \lipsum[1][3]
\item\label{rt:II} \lipsum[1][4]
\item\label{rt:III} \lipsum[1][5]
\item\label{rt:IV} \lipsum[1][6]
\item\label{rt:V} \lipsum[1][7]
\end{enumerate}
\end{researchtaskbox}

\begin{infobox}{Small Icon Info Box}
With additional infos.
\end{infobox}

\begin{examplebox}{Example Box}
With an example
\end{examplebox}

\begin{examplebox}{Single Line Info Box: \normalfont\mdseries With normal font following.}
\vspace*{-0.165cm}
\end{examplebox}

\paperbox{Www20-frontpage.jpg}{%
\textit{``Discovering Mathematical Objects of Interest --- A Study of Mathematical Notations''} by \textbf{Andr\'{e} Greiner-Petter}, Moritz Schubotz, Fabien M\"{u}ller, Corinna Bretinger, Howard S.~Cohl, Akiko Aizawa, and Bela Gipp. \textbf{In:} \textit{Proceedings of the Web Conference} (WWW), 2020.%
}{Chapter~\ref{ch:related-work} --- \cite{GreinerPetterSMB20}}

\begin{definitionbox}{Definition Box}[def:caption]
A nice definition box.
\end{definitionbox}

\begin{translationbox}{Translation Box}
Usually you want nice code box inside, see the next box as an example.
\end{translationbox}

\begin{translationbox}{Translation Box with Code}
\begin{code}[mytex]
P_n^{(\alpha,\beta)}(z) = \frac{(\alpha+1)_n}{n!} {}_2F_1\left(-n,1+\alpha+\beta+n;\alpha+1;\tfrac{1}{2}(1-z)\right)
\end{code}
\end{translationbox}

\begin{translationbox}{Translation Box with Custom Code}
\begin{customcode}[language=maple, mathescape=false]
diff( exp(z^2)*erfc(z), [z$(n)] ) = (-1)^(n)*(2)^(n)*factorial(n)*exp(z^2)*erfc(n, z)
\end{customcode} %$ %fool editor that the equation is "closed"
\vspace{-0.2cm}
\begin{footnotesize}
\vspace{-0.15cm}
\hfill Redundant parentheses removed to improve readability.
\end{footnotesize}
\end{translationbox}

\begin{translationbox}{Translation of Bailey’s Transformation of Very-Well-Poised ${}_{8}\phi_{7}$}
\begin{customcode}[language=mymathematica, basicstyle=\LSTfont, numbers=none, stepnumber=1, numbersep=6pt]
QHypergeometricPFQ[{a, q*(a)^(Divide[1,2]),-q*(a)^(Divide[1,2]),b,c,d,e,f},{(a)^(Divide[1,2]), -(a)^(Divide[1,2]),a*q/b,a*q/c,a*q/d,a*q/e,a*q/f},q, Divide[(a)^(2)*(q)^(2),b*c*d*e*f]] 
 == Divide[Product[QPochhammer[Part[{a*q,a*q/(d*e),a*q/(d*f),a*q/(e*f)},i],q, Infinity],{i,1, Length[{a*q,a*q/(d*e),a*q/(d*f),a*q/(e*f)}]}], Product[QPochhammer[Part[{a*q/d,a*q/e,a*q/f,a*q/(d*e*f)},i],q, Infinity],{i,1, Length[{a*q/d,a*q/e,a*q/f,a*q/(d*e*f)}]}]]* QHypergeometricPFQ[{a*q/(b*c),d,e,f},{a*q/b,a*q/c,d*e*f/a},q,q]
 + Divide[Product[QPochhammer[Part[{a*q,a*q/(b*c),d,e,f,(a)^(2)*(q)^(2)/(b*d*e*f),(a)^(2)*(q)^(2)/(c*d*e*f)},i],q, Infinity],{i,1, Length[{a*q,a*q/(b*c),d,e,f,(a)^(2)*(q)^(2)/(b*d*e*f),(a)^(2)*(q)^(2)/(c*d*e*f)}]}], Product[QPochhammer[Part[{a*q/b,a*q/c,a*q/d,a*q/e,a*q/f,(a)^(2)*(q)^(2)/(b*c*d*e*f),d*e*f/(a*q)},i],q, Infinity],{i,1, Length[{a*q/b,a*q/c,a*q/d,a*q/e, a*q/f,(a)^(2)*(q)^(2)/(b*c*d*e*f),d*e*f/(a*q)}]}]]
 * QHypergeometricPFQ[{a*q/(d*e),a*q/(d*f), a*q/(e*f),(a)^(2)*(q)^(2)/(b*c*d*e*f)},{(a)^(2)*(q)^(2)/(b*d*e*f),(a)^(2)*(q)^(2)/(c*d*e*f), a*(q)^(2)/(d*e*f)},q,q]
\end{customcode}
\vspace{-0.35cm}
\begin{footnotesize}
\hfill\rule{0.4\textwidth}{.4pt}

\vspace{-0.15cm}
\hfill Linebreaks are manually added to improve readability.
\end{footnotesize}
\end{translationbox}

\begin{codebox}{A Code Box}
\begin{code}[MathML]
<mrow>
  <msub>
    <mi>P</mi>
    <mi>n</mi>
  </msub>
  <mo>
    <!-- Invisible 
    Funct. Appl. 
    Unicode U+2061 -->  
  </mo>
  <mrow>
    <mo>(</mo>
    <mi>x</mi>
    <mo>)</mo>
  </mrow>
</mrow>
\end{code}
\end{codebox}

\begin{codebox}{A Code Box With Caption}[With Caption Text][code:caption]
\begin{code}[MathML]
<mrow></mrow>
\end{code}
\end{codebox}

\begin{codebox}{A Latex Code Box}[With caption.][lst:caption]
\begin{code}[mytex]
P_n^{(\alpha , \beta)}(x)          % Generic LaTeX
\JacobipolyP{n}{\alpha}{\beta}@{x} % Semantic LaTeX
\end{code}
\end{codebox}


\section{Example Section}\label{sec:research-gap}

\lipsum[3]

\section{Example Outline}\label{sec:outline}
\noindent\textbf{\Cref{ch:introduction}}
\lipsum[1][1-2]

\noindent\textbf{\Cref{ch:related-work}}
\lipsum[1][1-2]

\noindent\textbf{The \hyperref[ch:appendix]{Appendix}}
\lipsum[1][1-2]

\subsection{Example Overview of Publications}
\lipsum[2]

\begin{table}[ht]
\caption[Example of short caption for the List of Tables.]{Example full caption of a table.}
\label{tab:publications}
\centering
\renewcommand{\arraystretch}{1.1}
\begin{tabular}{c:l:l:l:l:c:l:r}\hline
	\textbf{Ch.} & \textbf{Venue} & \textbf{Year} & \textbf{Type} & \textbf{Length} & \makecell[bl]{\textbf{Author} \\\textbf{Position}} & \makecell[bl]{\textbf{Venue} \\\textbf{Rating}} & \textbf{Ref.} \\\hline
	\rowcolor{TableRowColor}\multirow{2}{*}{\cellcolor{white}\ref{ch:related-work}} 
	& \gls{scientometrics} & 2020 & Journal & Full & 1 of 7 & SJR Q1 & \cite{GreinerPetterYRM20} \\
	& \gls{www} & 2020 & Conference & Full & 1 of 7 & \glslink{core}{Core} A* & \cite{GreinerPetterSMB20} \\
 	\hline
\end{tabular}
\end{table}
